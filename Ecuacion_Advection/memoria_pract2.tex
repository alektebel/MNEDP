%%%%%%%%%%%%%%%%%%%%%%%%%%%%%%%%%%%%%%%%%%%%%%%%%%%%%%%%%%%%%%%%%%%%%%%%%%%
%
% Plantilla para un artículo en LaTeX en español.
%
%%%%%%%%%%%%%%%%%%%%%%%%%%%%%%%%%%%%%%%%%%%%%%%%%%%%%%%%%%%%%%%%%%%%%%%%%%%

\documentclass[10pt]{article}

% Esto es para que el LaTeX sepa que el texto está en español:
\usepackage[spanish]{babel}
\usepackage{amsmath,amsthm,amssymb}
\usepackage{graphicx}


%% Para añadir archivos con extensión pdf, jpg, png or tif
\usepackage{graphicx}


%% Primero escribimos el título
\title{Práctica 2 de MNEDP}
\author{Diego Rodríguez Atencia}

%% Después del "preámbulo", podemos empezar el documento

\begin{document}
\maketitle
\section{Introducción}
En las siguientes páginas, realizaremos un análisis de los métodos propuestos para aproximar numéricamente la ecuación planteada:\\
\begin{equation}
    u_{t} + a u_{x} = 0
\end{equation}
\section{La ecuación}
Esta se conoce como la ecuación de convección, útil en numerosos problemas donde tenemos masas de objetos que se mueven a cierta velocidad, como por ejemplo al meteorología, la hidráulica o la termodinámica. Aunque no sea especialmente complicada, es el entorno ideal para probar nuevos métodos numéricos para ecuaciones hiperbólicas, sin necesidad de aumentar mucho la comlejidad de los esquemas.\\
Primero, hayemos la solución, para observar la estabilidad y consistencia del esquema.\\
Podemos observar que existen funciones características en las que $u$ es constante.
Estas reciben el nombre de funciones características, y cumplen la siguiente ecuación:\\
\begin{equation}
    \frac{dx}{dt} = a(x, t)
\end{equation}
\\Entonces, cada función característica cumple la siguiente función:\\
\begin{equation}
    \frac{du}{dt} = \frac{\partial u}{\partial t} + \frac{\partial u}{\partial x}
    \frac{dx}{dt}
\end{equation}
De los datos iniciales tenemos que:\\
\begin{equation}
    u(x, 0) = u^{0}(x)
\end{equation}
Entonces, podemos encontrar la función característica que pasa por el punto $(x_{j}, 0)$ que sería precisamente $u(x, t) = u^{0}(x_{j})$.\\De esta forma, ya podemos resolver nuestra ecuación. Cabe destacar que para esta ecuación linear, las funciones características no se cruzan, mientras que $a$ sea Lipschitz-continua en $x$ y continua en $t$. En este caso, el $a(x, t)$ es el siguiente:
\begin{equation}
	a(x, t) = \frac{1+x^{2}}{1+2xt+ 2x^{2}+x^{4}}
\end{equation}
Como tanto $x$ como $t$ los podemos cubrir con un compacto suficientemente grande para nuestros experimentos, $a(x, t)$ es Lipschitz para el conjunto de los $(x, t) \in [0, 1]\times[0,M]$

\end{document} 